\subsection{Билет 27}

\textit{Самосопряжённое преобразование.}

\textbf{Опр. 1} Пусть $E $ - евкл. пр-во. Лин. преобр. $A : E \to E$ называется самосопр. (сс), если $A^* = A$

\textbf{Теорема 1 (Критерий)} $A : E \to E$ - лин. преобр., $E$ - евкл. пр., $0 < \mathrm{dim}\,E = n < \infty$

$A$-сс $\Leftrightarrow$ м-ца $[A]_e$ в любом базисе $e=(e_1,..,e_n)$ симметрична: $[A]_e' = [A]_e$

\textbf{Доказательство:} $A$ сс $\Leftrightarrow A^* = A \Leftrightarrow [ e$ - ортонорм. базис, $[A^*]_e = [A]_e' ] \Leftrightarrow$\\$ \Leftrightarrow [A^*]_e = [A]_e' \Leftrightarrow [A]_e = [A]_e' \Leftrightarrow $ симметр.

\textbf{Теорема 2} Пусть $A$-сс лин. преобр. $E$-ортонорм. Тогда все собственные значения преобразования $A$ вещественны.

\textbf{Доказательство:}


\textbf{Теорема 3} Собственные векторы сс. лин. преобр., соотв. различным собств. значениям, ортогональны

\textbf{Доказательство:}


\textbf{Теорема 4} Если подпр. $F$ инвариантно отн. $A$-сс преобр., то ортогональное доп-ние $F^\perp$ также инвар-но отн. $A\ (AF\subset F;\ AF^\perp \subset F^\perp)\ F^\perp = \{ x : (x,y)=0\ \forall y \in F \}$

\textbf{Доказательство:}


\textbf{Теорема 5 (основная)} Пусть $A$ - произв. сс. преобр. $E$

Тогда в $E\ \exists e=(e_1,..,e_n)$ - он базис из собств. векторов $A$. В Этом базисе $[A]_e$ имеет диаг. вид 
$\begin{pmatrix}
  \lambda_1 & \cdots & \mathcal{O} \\
  \cdots & \ddots & \cdots \\
  \mathcal{O} & \cdots & \lambda_n
 \end{pmatrix}
$;\quad $\lambda_k \in \mathbb{R};\ \lambda_k$ - собств. знач.

\textbf{Доказательство: }

\textbf{Опр. 2}
\textbf{Следствие из Т5} $\forall \mathcal{A}_{m\times n} \in M(\mathbb{R})$ - симм. $\exists Q_{n\times n}$ - ортогон: $(Q'\mathcal{A}Q)$ явл. диагональной матрицей.