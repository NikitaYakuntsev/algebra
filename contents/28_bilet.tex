\subsection{Билет 28}

\textit{Линейные и билинейные функции (формы)}

\begin{center}
 \textbf{Линейные формы}
\end{center}

\textbf{Опр.1} $E$ - лин. пр-во над полем $P$. \\
Линейное отображение $H : E \to P$ называется линейной формой $\Leftrightarrow \forall x,y \in E\ \forall \alpha \in P\ $
\begin{enumerate}
 \item $h(x+y) = h(x) + h(y)$
 \item $h(\alpha x) = \alpha h(x)$
\end{enumerate}

Пусть $\mathrm{dim}\, E = n, 0 < n < \infty;\ $ В $E$ выбран любой базис $(e_1, e_2,\ldots, e_n) = e;\ $ В $P$ выберем любой базис $f = (f_1 \ne 0), f_1 \in P\ (f_1 = 1)$\\

\textbf{Опр.2} Матрицей линейной формы $h$  в базисах $e,f$ - это матрица-строка $[h]_{e,f} = \begin{pmatrix} [h(e_1)]_f & \cdots & [h(e_n)]_f  \end{pmatrix}$\\
При $f_1=1\ [h]_e = \begin{pmatrix} h(e_1) & \cdots & h(e_n) \end{pmatrix}$\\

\textbf{Теорема 1} Пусть $E - n$-мерное линейное пространство. ($0 < n < \infty$) с любым базисом $e=(e_1, \ldots, e_n)$\\
Тогда $\forall h$ - лин. форма на $E$ справедлива формула
 
$\forall x = \sum\limits_{i=1}^n \alpha_i e_i\quad h(x) = \begin{pmatrix} \varphi_1 & \varphi_2 & \cdots & \varphi_n \end{pmatrix} \begin{pmatrix} \alpha_1 \\ \cdots \\ \alpha_n \end{pmatrix} (1)$\\
где $\varphi_i = h(e_i), i = \overline{1,n}$; 

$(1) \Leftrightarrow (1') \quad h(x) = [h]_e [x]_e,\ [x]_e = \begin{pmatrix} \alpha_1  \\ \cdots \\ \alpha_n \end{pmatrix}$

\textbf{Доказательство: } $x = \sum\limits_{i=1}^n \alpha_i e_i \ $применим $h:$

$h(x) = h\left( \sum\limits_{i=1}^n \alpha_i e_i \right) = \sum\limits_{i=1}^n \alpha_i \underset{:=\varphi_i}{h(e_i)} = \sum\limits_{i=1}^n \alpha_i \varphi_i = \begin{pmatrix} \varphi_1 & \cdots & \varphi_n \end{pmatrix} \begin{pmatrix} \alpha_1 \\ \cdots \\ \alpha_n \end{pmatrix}$
%end of th

\textbf{Теорема 2} $E - n-$мерное линейное пространство над полем $P; (0 < n < \infty); e=(e_1, \ldots, e_n)$ - произв. базис $E$

Тогда ф-ция, опр-ная $\forall x = \sum\limits_{i=1}^n \alpha_i e_i$ ф-лой $h(x) = \begin{pmatrix} \varphi_1 & \cdots & \varphi_n \end{pmatrix} \begin{pmatrix} \alpha_1 \\ \cdots \\ \alpha_n \end{pmatrix} = \Phi*[x]_e,\ $где $\begin{pmatrix} \varphi_1 & \cdots & \varphi_n \end{pmatrix}$ - любая матрица из $M_{1,n}(P),$ - 
\underline{линейна}, и ее матрица в базисе $e$ - это матрица $\begin{pmatrix} \varphi_1 & \cdots & \varphi_n \end{pmatrix}$

\textbf{Доказательство:} $\Phi = \begin{pmatrix} \varphi_1 & \cdots & \varphi_n \end{pmatrix}; [x]_e = \begin{pmatrix} \alpha_1 \\ \cdots \\ \alpha_n \end{pmatrix}\ \forall x = \sum\limits_{i=1}^n \alpha_i e_i $

\begin{enumerate}
 \item $\forall x,y \in E\ [x+y]_e = [x]_e + [y]_e;\ h(x+y) = \Phi[x+y]_e = \Phi([x]_e + [y]_e) = \Phi[x]_e + \Phi[y]_e =$\\$= h(x) + h(y)$
 \item $\forall \alpha \in P\ \forall x \in E\ [\alpha x]_e = \alpha[x]_e$\\
    $h(\alpha x) = \Phi[\alpha x]_e = \Phi(\alpha [x]_e) = \alpha \Phi[x]_e = \alpha h(x)$

    $h(e_1) = \begin{pmatrix} \varphi_1 & \cdots & \varphi_n \end{pmatrix} \begin{pmatrix} 1 \\ \cdots \\ 0 \end{pmatrix} = \varphi_1$\\
    $\ldots$\\
    $h(e_i) = \begin{pmatrix} \varphi_1 & \cdots & \varphi_n \end{pmatrix} \begin{pmatrix} 0 \\ \cdots \\ 1 \\ \cdots \\ 0 \end{pmatrix} i = \varphi_i $\\

    $[h]_e = \begin{pmatrix} h(e_1) & \cdots & h(e_i) & \cdots & h(e_n) \end{pmatrix} = \begin{pmatrix} \varphi_1 & \cdots & \varphi_n \end{pmatrix} = \Phi$
\end{enumerate}
%end of th

\textbf{Опр. 3} Пусть $E$ - лин. пр-во над полем $P$. Суммой двух лин. форм $h_i : E \to P,\ i=\overline{1,2},$\ называется ф-ция, опр. ф-лой $\forall x \in E\ h_1(x) + h_2(x)$

Произведением лин. формы $h : E \to P\ $ на число $\alpha \in P\ $ наз-ся ф-ция, опр. ф-лой $\forall x\in E\ \alpha h(x)$

\underline{Замечание:} в силу Т. для лин. операторов сумма и произведение являются линейной формой

\textbf{Доказательство:} %TODO добавляя доказательства, следи за переходом на новый лист в PDF!!
\\

\textbf{Теорема 3}
\begin{enumerate}
 \item \underline{Мн-во $E^*$} лин. форм на лин. пр-ве $E$ над полем $P$ \underline{является лин. пр-вом} над $P$
 \item Если $\mathrm{dim}\, E = n\ (0 < n <\infty), то \mathrm{dim}\, E^* = n$
\end{enumerate}

\textbf{Доказательство:}
\begin{enumerate}
 \item %TODO проверить 8 аксиом линейного пространства
 \item $\forall h \in E^*$ - лин. пр. $\overset{A}\leftrightarrow [h]_e$ - взаимно-однозначно в фикс. базисе $e=(e_1, \ldots, e_n)$

    $h_1 \to [h_1]_e,\ h_2\to [h_2]_e$ и $[h_1]_e = [h_2]_e \Rightarrow h_1 = h_2$

    $\forall h_1, h_2 \in E^*\ [h_1 + h_2]_e = [h_1]_e + [h_2]_e;\quad \forall \alpha \in P\ \forall h\in E^*\ [\alpha h]_e = \alpha [h]_e$

    $A(h) = [h]_e;\ A$-линейное и вз. однозн. $\Leftrightarrow$ биект. $\Rightarrow A$ - обр-мое $\Rightarrow A$ - изоморфизм $E^*$ и $M_{1,n}(P)$

    Базисом $M_{1,n}(P)$ явл-ся $q_1=(1,0,\ldots,0),\ q_2=(0,1,\ldots,0),\ldots,\ q_n = (0,0,\ldots, 1) \Rightarrow \mathrm{dim}\, M_{1,n} (P) = n \Rightarrow \mathrm{dim}\, E^* = n$
\end{enumerate}
%end of th

\textbf{Теорема 4} Если $E-n$-мерное лин. пр-во над полем $P$, то $(E^*)^*$ изоморфно $E$\\
\\
\\
\\
\\
\\
\\
\\
\\
\\
\\
\\ %mad dick
\\
\begin{center}
 \textbf{Билинейные формы}
\end{center}
Пусть $E$ - лин. пр-во над полем $P$

\textbf{Опр. 1} Билинейной формой наз-ся любое отображение $b : E \times E \to P\ $ \\
(т.е. $\forall x,y\in E\ \exists b(x,y)\in P$; линейно по $x$ при $\forall y$ - фикс.; линейно по $y$ при $\forall x$ - фикс.

\textbf{Свойства:}
\begin{enumerate}
 \item $\forall x,y,z \in E\ b(x+y,z) = b(x,z) + b(y,z)$
 \item $\forall x,y \in E\ \forall \alpha \in P\ b(\alpha x,y) = \alpha b(x,y)$
 \item $\forall x,y,z \in E\ b(x,y+z) = b(x,y) + b(x,z)$
 \item $\forall x,y \in E\ \forall \alpha \in P b(x,\alpha y) = \alpha b(x,y)$
\end{enumerate}

\textbf{Опр. 2} Матрицей билин. формы $b(x,y)$ в базисе $e=(e_1,\ldots,e_n)$ называется матрица
 $[b]_e = \begin{pmatrix}
           b(e_1, e_1) & \cdots & b(e_1, e_n) \\
	   \cdots & \ddots & \cdots \\
	   b(e_n, e_1) & \cdots & b(e_n, e_n)
          \end{pmatrix}\quad b_{i,j} = b(e_i, e_j)$\\

\textbf{Теорема 1.1} Пусть $b(x,y)$ - билин. форма на $E$ с матрицей $[b]_e = (b_{i,j})_{i,j = \overline{1,n}}$

Тогда $\forall x,y \in E\ x = \sum\limits_{i=1}^n \alpha_i e_i\ y = \sum\limits_{j=1}^n \beta_j e_j;\ $

$b(x,y) = \sum\limits_{i=1}^n \sum\limits_{j=1}^n b_{i,j} \alpha_i \beta_j\ (1) \Leftrightarrow b(x,y) = [x]_e' [b]_e [y]_e\ (2)$

$[x]_e' = \begin{pmatrix} \alpha_1 & \cdots & \alpha_n \end{pmatrix}\quad [y]_e =\begin{pmatrix} \beta_1 \\ \cdots \\ \beta_n \end{pmatrix}\ $

\textbf{Доказательство:} $b(x,y) = b\left(\sum\limits_{i=1}^n \alpha_i e_i,\ \sum\limits_{j=1}^n \beta_j e_j\right) = \sum\limits_{i=1}^n \sum\limits_{j=1}^n b(e_i, e_j) \alpha_i \beta_j\ = \sum\limits_{i=1}^n \sum\limits_{j=1}^n b_{i,j} \alpha_i \beta_j\ $

$\begin{pmatrix}\alpha_1 & \cdots & \alpha_n \end{pmatrix} \begin{pmatrix} b_{1,1} & \cdots & b_{1.n} \\ \cdots & \ddots & \cdots \\ b_{n,1} & \cdots & b_{n,n} \end{pmatrix} \begin{pmatrix} \beta_1 \\ \cdots \\ \beta_n \end{pmatrix} =
\begin{pmatrix}\alpha_1 & \cdots & \alpha_n \end{pmatrix}
\begin{pmatrix}
 \sum\limits_{j=1}^n b_{1,j} \beta_j \\
 \cdots \\
 \sum\limits_{j=1}^n b_{n,j} \beta_j
\end{pmatrix} = \sum\limits_{i=1}^n \sum\limits_{j=1}^n b(e_i, e_j) \alpha_i \beta_j
$\\

\textbf{Теорема 1.2} Для любой матрицы $\mathcal{B} = (b_{i,j})_{i,j=\overline{1,n}} $ функция $b(x,y)$, определенная для $\forall x = \sum\limits_{i=1}^n \alpha_i e_i;\ \forall y = \sum\limits_{j=1}^n \beta_j e_j$ формулой
$b(x,y) = \sum\limits_{i=1}^n \sum\limits_{j=1}^n b_{i,j} \alpha_i \beta_j\ (3) \Leftrightarrow b(x,y) = [x]_e' \mathcal{B} [y]_e\ (4)$, является билинейной, а ее матрица в базисе $e$ совпадает с матрицей $\mathcal{B}$

\textbf{Доказательство: } Траспонируем $(4): b(x,y) = \left( [x]_e' \mathcal{B} [y]_e \right)' = \left( [y]_e' \mathcal{B} \right)[x]_e = (z_1\ldots z_n)[x]_e$ - лин. по $x\ \forall y$ - фикс.

$b(x,y) = \left( [x]_e' \mathcal{B} \right)[y]_e = (u_1\ldots u_n) [y]_e$ - лин. по $y\ \forall x$ - фикс.

М-ца $b(x,y)$ в базисе $e$ совп. с $\mathcal{B};\ b(e_i, e_j) = \underset{i} {\begin{pmatrix} 0 & \cdots & 1 & \cdots & 0 \end{pmatrix}} \mathcal{B} \begin{pmatrix} 0 \\ \cdots \\ 1 \\ \cdots \\ 0 \end{pmatrix}{}_j = b_{i,j}$

Матрица $\left( b(e_i, e_j) \right)_{i,j=\overline{1,n}} = \mathcal{B}$\\
%end of th

\textbf{Опр. 3} Билин. форма $b(x,y)$ на $E$ называется симметричной, если $\forall x,y \in E\ b(x,y) = b(y,x)$

\textbf{Теорема 3 (симметрия)} Билинейная форма на $n$-мерном пространстве симм. $\Leftrightarrow$ \underline{матрица} билин. формы в некотором базисе ($\Rightarrow$ в $\forall$) \underline{симметрична}

\textbf{Доказательство:}
\begin{enumerate}
 \item билинейная форма симм $\overset{def} \Leftrightarrow b(x,y) = b(y,x)\ \forall x,y \in E$

    $b(e_i, e_j) = b(e_j, e_i)\ \forall i,j = \overline{1,n}\ \forall e=(e_1,\ldots,e_n) \Rightarrow$ матрица симметрична

 \item Пусть матр. билин. формы $b(x,y)$ симметрична в некотором базисе $e=(e_1,\ldots, e_n) \Rightarrow b(e_i, e_j) = b(e_j, e_i)$

    $b(x,y): x = \sum\limits_{i=1}^n \alpha_i e_i;\ y = \sum\limits_{j=1}^n \beta_j e_j;\quad b(x,y) = [x]_e' [b]_e [y]_e$ трансп. $\Rightarrow $

    $b(x,y = [y]_e' [b]_e' [x]_e;\ b(y,x) = [y]_e' [b]_e [x]_e);\ $т.к. м-ца симметр., то $[b]_e' = [b]_e \Rightarrow b(x,y) = b(y,x)\ \forall x,y\in E$

    исп. 1, получим что $b(x,y)$ симм. в $\forall$ базисе.
\end{enumerate}
%end of th

\textbf{Теорема 4 (изм. матрицы билин. формы при изм. базиса)}

При изменении базиса $e=(e_1,\ldots,e_n) \to e' = (e_1',\ldots, e_n')$ с матр. перех. $T_{e\to e'}$ матрицы билин. формы в разных базисах связаны формулой $[b]_{e'} = T_{e\to e'}' [b]_e T_{e\to e'}$

\textbf{Доказательство: } $b(x,y) = [x]_e' [b]_e [y]_e;\quad [y]_e = T_{e\to e'}[y]_{e'};\ [x]_e = T_{e\to e'} [x]_{e'}; $ подставляем

$b(x,y) = \left( T_{e\to e'} [x]_{e'} \right)' [b]_e \left( T_{e\to e'}[y]_{e'} \right) = [x]_{e'}' \left( T_{e\to e'}' [b]_e T_{e\to e'} \right) [y]_{e'};$

$\Bigg\{ \begin{matrix} b(x,y) = [x]_{e'}' \left( T_{e\to e'}' [b]_e T_{e\to e'} \right) [y]_{e'} \\ b(x,y) = [x]_{e'}' [b]_{e'} [y]_{e'} \end{matrix} $ 

$[x]_{e'}' [b]_{e'} [y]_{e'} = [x]_{e'}' \left( T_{e\to e'}' [b]_e T_{e\to e'} \right) [y]_{e'} \Rightarrow [b]_{e'} = T_{e\to e'}' [b]_e T_{e\to e'}$

  


