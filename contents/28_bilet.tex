\subsection{Билет 28}

\textit{Линейные и билинейные функции (формы)}

\textbf{Опр.1} $E$ - лин. пр-во над полем $P$. \\
Линейное отображение $H : E \to P$ называется линейной формой $\Leftrightarrow \forall x,y \in E\ \forall \alpha \in P\ $
\begin{enumerate}
 \item $h(x+y) = h(x) + h(y)$
 \item $h(\alpha x) = \alpha h(x)$
\end{enumerate}

Пусть $\mathrm{dim}\, E = n, 0 < n < \infty;\ $ В $E$ выбран любой базис $(e_1, e_2,\ldots, e_n) = e;\ $ В $P$ выберем любой базис $f = (f_1 \ne 0), f_1 \in P\ (f_1 = 1)$\\

\textbf{Опр.2} Матрицей линейной формы $h$  в базисах $e,f$ - это матрица-строка $[h]_{e,f} = \begin{pmatrix} [h(e_1)]_f & \cdots & [h(e_n)]_f  \end{pmatrix}$\\
При $f_1=1\ [h]_e = \begin{pmatrix} h(e_1) & \cdots & h(e_n) \end{pmatrix}$\\

\textbf{Теорема 1} Пусть $E - n$-мерное линейное пространство. ($0 < n < \infty$) с любым базисом $e=(e_1, \ldots, e_n)$\\
Тогда $\forall h$ - лин. форма на $E$ справедлива формула
 
$\forall x = \sum\limits_{i=1}^n \alpha_i e_i\quad h(x) = \begin{pmatrix} \varphi_1 & \varphi_2 & \cdots & \varphi_n \end{pmatrix} \begin{pmatrix} \alpha_1 \\ \cdots \\ \alpha_n \end{pmatrix} (1)$\\
где $\varphi_i = h(e_i), i = \overline{1,n}$; 

$(1) \Leftrightarrow (1') \quad h(x) = [h]_e [x]_e,\ [x]_e = \begin{pmatrix} \alpha_1  \\ \cdots \\ \alpha_n \end{pmatrix}$

\textbf{Доказательство: } $x = \sum\limits_{i=1}^n \alpha_i e_i \ $применим $h:$

$h(x) = h\left( \sum\limits_{i=1}^n \alpha_i e_i \right) = \sum\limits_{i=1}^n \alpha_i \underset{:=\varphi_i}{h(e_i)} = \sum\limits_{i=1}^n \alpha_i \varphi_i = \begin{pmatrix} \varphi_1 & \cdots & \varphi_n \end{pmatrix} \begin{pmatrix} \alpha_1 \\ \cdots \\ \alpha_n \end{pmatrix}$
%end of th

\textbf{Теорема 2} $E - n-$мерное линейное пространство над полем $P; (0 < n < \infty); e=(e_1, \ldots, e_n)$ - произв. базис $E$

Тогда ф-ция, опр-ная $\forall x = \sum\limits_{i=1}^n \alpha_i e_i$ ф-лой $h(x) = \begin{pmatrix} \varphi_1 & \cdots & \varphi_n \end{pmatrix} \begin{pmatrix} \alpha_1 \\ \cdots \\ \alpha_n \end{pmatrix} = \Phi*[x]_e,\ $где $\begin{pmatrix} \varphi_1 & \cdots & \varphi_n \end{pmatrix}$ - любая матрица из $M_{1,n}(P),$ - 
\underline{линейна}, и ее матрица в базисе $e$ - это матрица $\begin{pmatrix} \varphi_1 & \cdots & \varphi_n \end{pmatrix}$

\textbf{Доказательство:} $\Phi = \begin{pmatrix} \varphi_1 & \cdots & \varphi_n \end{pmatrix}; [x]_e = \begin{pmatrix} \alpha_1 \\ \cdots \\ \alpha_n \end{pmatrix}\ \forall x = \sum\limits_{i=1}^n \alpha_i e_i $

\begin{enumerate}
 \item $\forall x,y \in E\ [x+y]_e = [x]_e + [y]_e;\ h(x+y) = \Phi[x+y]_e = \Phi([x]_e + [y]_e) = \Phi[x]_e + \Phi[y]_e =$\\$= h(x) + h(y)$
 \item $\forall \alpha \in P\ \forall x \in E\ [\alpha x]_e = \alpha[x]_e$\\
$h(\alpha x) = \Phi[\alpha x]_e = \Phi(\alpha [x]_e) = \alpha \Phi[x]_e = \alpha h(x)$

$h(e_1) = \begin{pmatrix} \varphi_1 & \cdots & \varphi_n \end{pmatrix} \begin{pmatrix} 1 \\ \cdots \\ 0 \end{pmatrix} = \varphi_1$\\
$\ldots$\\
$h(e_i) = \begin{pmatrix} \varphi_1 & \cdots & \varphi_n \end{pmatrix} \begin{pmatrix} 0 \\ \cdots \\ 1 \\ \cdots \\ 0 \end{pmatrix} i = \varphi_i $\\

$[h]_e = \begin{pmatrix} h(e_1) & \cdots & h(e_i) & \cdots & h(e_n) \end{pmatrix} = \begin{pmatrix} \varphi_1 & \cdots & \varphi_n \end{pmatrix} = \Phi$
\end{enumerate}
%end of th

\textbf{Опр. 3} Пусть $E$ - лин. пр-во над полем $P$. Суммой двух лин. форм $h_i : E \to P,\ i=\overline{1,2},$\ называется ф-ция, опр. ф-лой $\forall x \in E\ h_1(x) + h_2(x)$

Произведением лин. формы $h : E \to P\ $ на число $\alpha \in P\ $ наз-ся ф-ция, опр. ф-лой $\forall x\in E\ \alpha h(x)$

\underline{Замечание:} в силу Т. для лин. операторов сумма и произведение являются линейной формой

\textbf{Доказательство:} %TODO
\\

\textbf{Теорема 3}
\begin{enumerate}
 \item \underline{Мн-во $E^*$} лин. форм на лин. пр-ве $E$ над полем $P$ \underline{является лин. пр-вом} над $P$
 \item Если $\mathrm{dim}\, E = n\ (0 < n <\infty), то \mathrm{dim}\, E^* = n$
\end{enumerate}

\textbf{Доказательство:}
\begin{enumerate}
 \item %TODO проверить 8 аксиом линейного пространства
 \item $\forall h \in E^*$ - лин. пр. $\overset{A}\leftrightarrow [h]_e$ - взаимно-однозначно в фикс. базисе $e=(e_1, \ldots, e_n)$

$h_1 \to [h_1]_e,\ h_2\to [h_2]_e$ и $[h_1]_e = [h_2]_e \Rightarrow h_1 = h_2$

$\forall h_1, h_2 \in E^*\ [h_1 + h_2]_e = [h_1]_e + [h_2]_e;\quad \forall \alpha \in P\ \forall h\in E^*\ [\alpha h]_e = \alpha [h]_e$

$A(h) = [h]_e;\ A$-линейное и вз. однозн. $\Leftrightarrow$ биект. $\Rightarrow A$ - обр-мое $\Rightarrow A$ - изоморфизм $E^*$ и $M_{1,n}(P)$

Базисом $M_{1,n}(P)$ явл-ся $q_1=(1,0,\ldots,0),\ q_2=(0,1,\ldots,0),\ldots,\ q_n = (0,0,\ldots, 1) \Rightarrow \mathrm{dim}\, M_{1,n} (P) = n \Rightarrow \mathrm{dim}\, E^* = n$
\end{enumerate}
%end of th

\textbf{Теорема 4} Если $E-n$-мерное лин. пр-во над полем $P$, то $(E^*)^*$ изоморфно $E$


 









