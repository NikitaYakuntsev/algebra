\section{Функции многих переменных}
\subsection{Билет 19}


\textit{Евклидово пространство. Простейшие свойства.}


 $E$ - вещественное линейное пространство называется евклидовым, если задана функция $E \times E \to \mathbb{R}$, называемая скалярным произведением \\ 
(обозн. $\forall x, y\ \exists (x, y) \in \mathbb{R}$) и выполнено 4 аксиомы:
  \begin{enumerate}
    \item $\forall x, y \in E\ (x, y) = (y, x)$
    \item $\forall x_1, x_2, y\,\in E\ (x_1 + x_2, y) = (x_1, y) + (x_2, y) $
    \item $\forall \alpha\,\in \mathbb{R}\ \forall x, y\,\in E\ (\alpha x, y) = \alpha(x, y)$
    \item $\forall x \in E\ (x, x) \ge 0, x = 0 \Leftrightarrow (x, x) = 0$
  \end{enumerate}
  
  \begin{center}
   \textbf{Свойства:}
  \end{center}



