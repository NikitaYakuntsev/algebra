\subsection{Билет 22}


\textit{Ортогональность. Ортогональные системы векторов. Ортогонализация.}


\textbf{Def1: } Пусть $E$ - евклидово пространство
\begin{description}
 \item [a)] Векторы $x,y \in E\ $ - называются ортогональными $x \perp y$, если $\left( x,y \right)=0$
 \item [b)] $x \in E\ , M \subset E$ \\
	Вектор $x$ ортогонален множеству M, если $\forall y \in M\ , \ x \perp y \Leftrightarrow \forall y \in M\ \left( x,y \right) = 0 $ \\
	Обозначение: $x \perp M$ 
 \item [c)] $M,P \subset E$ \\
	Множества $M,P$ - ортогональны $\left( M \perp P \right),$ если $\forall x \in M\ , y \in P\, \ \left( x,y \right)=0  $ \\
	$x \perp E \Leftrightarrow x=0 $
\end{description} 
\textbf{Теорема 1} \\
 Пусть $x \perp M \Leftrightarrow$ для любого базиса $(e_1,e_k)$ в $M,  x \perp e_i,  i=1..k,  \Leftrightarrow (x,e_i)=0, i=1..k $ \\
\textbf{Доказательство:} \\
\begin{description}
 \item [$\Rightarrow :$] $x \perp M \Leftrightarrow \forall y \in M\, \ x \perp y, \ \Leftrightarrow (x,y)=0 \Rightarrow$ верно и для базисных 
	$(e_1..e_k), \ y=e_i, \ i=1..k$
 \item [$\Leftarrow :$] $x \perp e_i, \ i=1..k, \ \forall y \in M\ $ разложим по базису $y = \sum \limits_{i=1}^k {\alpha_i e_i}$ \\
	$x \perp e_i \Rightarrow (x,e_i)=0, \ \forall i=1..k$ \\
	$(x,y)=(x,\sum \limits_{i=1}^k {\alpha_i,e_i}) = \sum \limits_{i=1}^k {\alpha_i (x,e_i) = 0, \ \forall y \in M\ \ (x,y)=0 \Rightarrow x \perp M } $
\end{description}
\underline{Замечание:} Если $M=L(e_1..e_k), x \perp e_i \ i=1..k \Rightarrow x \perp M $ \\
\textbf{Def2: } 
\begin{enumerate}
 \item Система векторов $\{x_{\alpha}\}$ из евклидового простраства называется ортогональной, если $x_{\beta} \perp x_{\alpha} \ forall \beta \ne \alpha$
 \item Система векторов $\{x_{\alpha}\}$ из евклидового простраства называется нормированной, если $(x_{\alpha},x_{\alpha}=1, \ \forall \alpha)$
 \item Система векторов $\{x_{\alpha}\}$ из евклидового простраства называется ортонормированной, если $1) \cap 2)$
\end{enumerate}
\textbf{Теорема 2} \\
Ортогональная система из ненулевых векторов линейно независимая \\
\textbf{Доказательство: } \\
$\{ x_{\alpha} \}$ - ортогональная система. Докажем, что и конечная подсистема $x_{\alpha_1}..x_{\alpha_k}$ - ЛНЗ \\
Рассмотрим линейную комбинацию $\sum \limits_{i=1}^k {\beta_i x_{\alpha_i}}=0; \ \beta_i=0-? \ i=1..k$ \\
Скалярно умножим $\sum \limits_{i=1}^k {\beta_i x_{\alpha_i}}=0$ на вектор $x_{\alpha_j} \ \forall j=1..k$ \\
$\left( \sum \limits_{i=1}^k {\beta_i x_{\alpha_i}},x_{\alpha_j} \right)  =(0,x_{\alpha_j}) = 0$ \\
$\sum \limits_{i=1}^k {\beta_i (x_{\alpha_i}),x_{\alpha_j})} = 0 \Rightarrow \beta_j (x_{\alpha_j}) = 0 $ \\ 
$x_{\alpha_j} = $
$\left\{ \begin{matrix}
\mbox{$0, \ i \ne j $} \\
\mbox{$(x_{\alpha_j},x_{\alpha_j}), \ i=j$}
\end{matrix}\right.$
$(x_{\alpha_j},x_{\alpha_j}) \ne 0 \Rightarrow \beta_j = 0 \Rightarrow$ все $\beta_j=0, \ j=1..k \Rightarrow$ система ЛНЗ \\
\textbf{Теорема 3(Метод Фурье)} \\
Пусть $e_1..e_n$ ортогональная система из ненулевых векторов и $x=\sum \limits_{i=1}^n {\alpha_i e_i} (*) $ \\
Тогда $\alpha_i= \frac{(x,e_i)}{e_i,e_i}, \ i=1..n$\\
$\alpha_i$- коээфициенты Фурье (*)- формула Фурье \\
\textbf{Доказательство: } \\
Умножим (*) скалярно на $e_j$ \\
$(x,e_j)=\left( \sum \limits_{i=1}^{n} {\alpha_i e_i},e_j \right), \ (x,e_j) = \sum \limits_{i=1}^n {\alpha_i (e_i,e_j)} $ \\
$(e_i,e_j)=\left\{ \begin{matrix}
\mbox{$0, \ i \ne j $} \\
\mbox{$(e_j,e_j), \ i=j$}
\end{matrix}\right.$
$(x,e_j)=\alpha_j(e_j,e_j), \ \Bigl| :(e_j,e_j) \ne 0 $ \\
$\alpha_j= \frac{(x,e_j)}{e_i,e_j}$ \\
\textbf{Следствия} \\
\begin{enumerate}
 \item $x=\sum \limits_{i=1}^n {\frac{(x,e_i)}{e_i,e_i} e_i}$
 \item Пусть $(e_1..e_n)$ - ортонормированная система $ x = \sum \limits_{i=1}^n {\alpha_i e_i} \Rightarrow \alpha_i = (x,e_i) \ i=1..n $ \\
       $x=\sum \limits_{i=1}^n {(x,e_i)e_i} $
\end{enumerate}
\textbf{Доказательство следствия 2:} \\
$(e_i,e_j)=\left\{ \begin{matrix}
\mbox{$0, \ i \ne j $} \\
\mbox{$1, \ i=j$}
\end{matrix}\right. \Rightarrow (e_i,e_i)=1 \Rightarrow e_i \ne 0 $ \\
$\alpha_i = \frac{(x,e_i)}{(e_i,e_i)}=(x,e_i)$ \\
\textbf{Теорема 3(Грама-Шмидта)} \\
Пусть $e_1..e_n$ ЛНЗ система в евклидовом пр-ве E. Тогда существует ортогональная система $(f_1..f_n)$ для которых 
$L(e_1..e_k)=L(f_1..f_k), \forall k =1..n$ \\
\textbf{Доказательство:} \\
1) $f_1=\frac{e_1}{\left| \left| e_1 \right| \right|}$ - вектор единичной длины \\
$(f_1,f_1)=\left(\frac{e_1}{\left| \left| e_1 \right| \right|},\frac{e_1}{\left| \left| e_1 \right| \right|} \right) = 1 \frac{1}{\left| \left| e_1 \right| \right|^2}(e_1,e_1)=
\frac{\left| \left| e_1 \right| \right|^2}{\left| \left| e_1 \right| \right|^2} = 1$ \\
$L(e_1)=L(f_1), \ f_1=\alpha e_1 , \ \alpha = \frac{1}{\left| \left| e_1 \right| \right|}$ \\
2)Ищем $e_2'=\alpha_1 f_1 + e_2$($\alpha_1$-неизв), так чтобы $e_2' \perp f_1, \ (e_2',f_1)=0$ \\
$(\alpha_1 f_1 + e_2,f_1)=0, \ \alpha_1(f_1,f_1)+(e_2,f_1)=0$, т.к. $(f_1,f_1)=1 \Rightarrow \alpha_1=-(e_2,f_1) \Rightarrow e_2'=-(e_2,f_1)f_1+e_2$ \\
$f_2=\frac{f_2'}{\left| \left| e_2' \right| \right|}, \ e_2' \ne 0 ?$ \\
\underline{Предположим противное} $e_2'=0, \ \alpha_1 f_1 + e_2 0 , \ f_1=\frac{e_1}{\left| \left| e_1 \right| \right|}$, но $e_1,e_2$-ЛНЗ \\
3)Пусть для $m$ векторов построены $f_1,,f_m$ - удовл условию Т3 \\
4)При m<n ищем вектор $f_{m+1}, \ e_{m+1}' = \alpha_1 e_1 + .. + \alpha_m f_m + e_{m+1}$, так чтобы $e_{m+1}'$ был ортогонален $f_i, \ i=1..m, \ 
e_{m+1}' \perp f_i \Leftrightarrow (e_{m+1}',f_i)=0$ \\
$(e_{m+1}',f_i)=0 \Leftrightarrow \alpha_1 \underset {=1} {(f_1,f_1)} + \alpha_2 \underset {=0} {(f_2,f_1)} + .. + \alpha_2 \underset {=0} {(f_m,f_1)} + \alpha_m \underset {=0} {(e_{m+1},f_1)}$ \\
.........\\
$(e_{m+1}',f_m)=0 \Leftrightarrow \alpha_1 \underset {=0} {(f_1,f_m)} + \alpha_2 \underset {=0} {(f_2,f_m)} + .. + \alpha_m \underset {=1} {(f_m,f_m)} + \alpha_2 \underset {=0} {(e_{m+1},f_m)}$ \\
$(f_i,f_j) = \left\{ \begin{matrix}
\mbox{$0, \ i \ne j $} \\
\mbox{$1, \ i=j$}
\end{matrix}\right.$ \\
$\alpha_1=-(e_{m+1},f_1)$ \\
.......... \\
$\alpha_m=-(e_{m+1},f_m)$ \\
(*) $e_{m+1}'=\alpha_1 f_1 + .. + \alpha m f_m + e_{m+1}, \ \alpha_i=-(e_{m+1},f_i), \ i=1..m$ \\
$e_{m+1}' \ne 0 $ \underline{Предположим противное} $e_{m+1}'=0 \Rightarrow f_1..f_m$-выражаются через $(e_1..e_m)$ \\
получается нетривиальная линейная комбинация $(e_1..e_{n+1})$ \\
$f_{m+1}=\frac{e_{m+1}'}{\left| \left| e_{m+1}' \right| \right|} \Rightarrow \left| \left| f_{m+1} \right| \right| =1 $ \\
Построим ортогональную систему $ (*) \Rightarrow f_{m+1}$ - линейно выражается через $(e_1..e_{m+1})$ \\ 