\subsection{Билет 29}

\textit{Квадратичные функции (формы)}\\

\textbf{Опр. 1} Квадр. формой на $E$ - лин. пр-во над полем $P$ называется ф-ция, определяемая равенством $k(x)=b(x,x)\ \forall x\in E$, где $b(x,y)$ - симм. билин. форма на $E$ \\

\textbf{Теорема 1} Для заданной кв. формы $k(x)$ на $E \exists! b(x,y)$ - симм. билин. форма, для которой $k(x) = b(x,x)\ \forall x\in E$

\textbf{Доказательство: } $b_1{x,y}, b_2(x,y)$ - симм. билин.;\quad $k(x) = b_1(x,x); k(x) = b_2(x,x)$

$k(x+y) = b_1(x+y, x+y) = b_1(x,x) + b_1(x,y) + b_1(y,x) + b_1(y,y) = b_1(x,x) +  2b_1(x,y) +$\\$+ b_1(y,y) = k(x)+2b_1(x,y)+k(y) \Rightarrow b_1(x,y) = \frac{k(x+y) - k(x)-k(y)}{2}$

$b_2 = \frac{k(x+y) - k(x)-k(y)}{2} \Rightarrow b_1(x,y) = b_2(x,y)\ \forall x,y \in E$\\

\underline{Следствие:} по заданной кв. форме восстанавливается симм. билин. форма

\textbf{Опр. 2} М-цей кв. формы $k(x) = b(x,x)$ на $E$ наз-ся матрица симм. билин. формы $b(x,y)$

\textbf{Опр. 3} Квадр. форма $k(x) = b(x,x)$ на конечном лин. пр-ве имеет канонический вид в базисе $e$, если ее матрица в базисе имеет диаг. вид $[b]_e = \begin{pmatrix} \lambda_1 & \mathcal{O} \\ \cdots & \cdots \\ \mathcal{O} & \lambda_n  \end{pmatrix}$\\
Этот базис называется каноническим и кв. форма имеет вид $\textbf{(*)}\ k(x) = \lambda_1 \alpha_1^2 + \ldots + \lambda_n \alpha_n^2,\ $\\ где $x = \sum\limits_{i=1}^n \alpha_i e_i$\\

\textbf{Теорема 2} $\forall k(x) = b(x,x)$ - кв. форма на конечном $n-$мерном евкл. пр-ве $E \exists$ канонический базис $e=(e_1,\ldots,e_n)$, в котором кв. форма имеет канонический вид \textbf{(*)}

\textbf{Доказательство: } Метод Лагранжа (выделение квадратов): $x = \sum\limits_{i=1}^n \alpha_i e_i$

$k(x) = b(x,x) = \sum\limits_{i,j=1}^n b(e_i, e_j) \alpha_i \alpha_j = \sum\limits_{i,j=1}^n b_{i,j} \alpha_i \alpha_j (**)$

Индукция по кол-ву $m$ координат $\alpha_1,\ldots,\alpha_n$, входящих в ф-лу (**) с ненуленвыми коэф-тами:
\begin{enumerate}
 \item $m=1$, без огр. общности $\alpha_1;\quad k(x) = b_{1,1} \alpha_1^2$ верно
 \item при $\ \forall m < p$ утв. верно: если кв. форма содержит $\alpha_1,..,\alpha_m$ с ненул. коэф., то\\ $\exists e' = (e_1',..,e_n')$ - базис, в котором м-ца имеет вид
$\begin{pmatrix}
  b_{1,1}' & \cdots & \cdots & \cdots & \mathcal{O} \\
  \mathcal{O} & b_{2,2}' & \cdots & \cdots & \mathcal{O} \\
  \cdots & \cdots & \ddots & \cdots & \cdots \\
  \mathcal{O} & \cdots & \cdots & b_{m,m}' & \mathcal{O} \\
  \mathcal{O} & \mathcal{O} & \cdots & \mathcal{O} & \ddots
 \end{pmatrix}
$
 \item Док. для $m=p$

$b(x,x) = \sum\limits_{i,j=1}^p b_{i,j} \alpha_i \alpha_j = | b_{i,j} = b_{j,i} | = \sum\limits_{i,j=1}^n b_{i,i}^2 + 2 \sum\limits_{1 \le i < j \le p} b_{i,j} \alpha_i \alpha_j$

Если все $b_{i,j}=0$ при $i<j$, то кв. форма уже имеет канонический вид.
Пусть не все $b_{i,j}=0$
\begin{enumerate}
 \item Пусть все $b_{i,i}=0,\ i=\overline{1,p};\ $ Если $b_{i,j} \ne 0, i<j$ то делаем замену:

    $\alpha_i = \alpha_i' - \alpha_j';\ \alpha_j = \alpha_i' + \alpha_j';\ \alpha_k = \alpha_k';\quad \forall k \ne j, k\ne i$

    $2b_{i,j}\alpha_i \alpha_j = \underset{\ne 0} {2b_{i,j}} (\alpha_i')^2 - \underset{\ne 0}{2b_{i,j}} (\alpha_j')^2$

    $\begin{pmatrix}
      \alpha_1 \\ \cdots \\
      \alpha_i \\ \cdots \\
      \alpha_j \\ \cdots \\
      \alpha_n 
     \end{pmatrix}  = \{T: m_{l,l} = 1; m_{i,i} = 1; m_{i,j} = -1; m_{j,i} = 1; m_{j,j} = 1; i < j\} * 
     \begin{pmatrix} 
      \alpha_1' \\ \cdots \\
      \alpha_i' \\ \cdots \\
      \alpha_j' \\ \cdots \\
      \alpha_n' 
     \end{pmatrix}
    $\quad $\left| T \right| \ne 0;\ $ Матрица $T$ определяет переход от одного базиса к другому.
 \item Пусть в кв. форме $\exists i\ : b_{i,i} \ne 0. \alpha_i = \alpha_{p'}; \alpha p = \alpha_{i'}$ после замены коэф. при $(\alpha_p')^2 \ne 0$

$b(x,x) = \sum\limits_{i=1}^p b_{i,i} \alpha_i^2 + 2\sum\limits_{1\le i<j \le p} b_{i,j} \alpha_i \alpha_j = \underset{=c(\alpha_1,..,\alpha_{p-1})}{\left( \sum\limits_{i=1}^{p-1} b_{i,i} \alpha_i^2 + 2\sum\limits_{1\le i<j \le p-1} b_{i,j} \alpha_i \alpha_j \right)} + $\\
$+ \left( b_{pp} \alpha_p^2 + 2\sum\limits_{1\le i \le p-1} b_{i,p} \alpha_i \alpha_p \right) = c + b_{p,p}\left( \alpha_p^2 + 2\sum\limits_{1\le i \le p-1} \frac{b_{i,p}}{b_{p,p}} \alpha_i \alpha_p \right) = (\times)$  

добавим и вычтем $b_{p,p} \left( \sum\limits_{i=1}^{p-1} \left( \frac{b_{i,p}}{b_{p,p}}\alpha_i \right)^2 + 2 \sum\limits_{1 \le i < j \le p-1} \frac{b_{i,p} b_{j,p}}{b_{p,p}^2}\alpha_i \alpha_j \right) = d(\alpha_1,..,\alpha_{p-1})$

$(\times) = c + b_{p,p} \underset{=\alpha_p'}{\left( \alpha_p + \frac{b_{1,p}}{b_{p,p}}\alpha_1 + \ldots + \frac{b_{p-1,p}}{b_{p,p}}\alpha_{p-1} \right)^2} - d = (\times \times);\quad h = (c-d)$ - кв. форма, отн. $\alpha_1,..,\alpha_{p-1};\ m = p-1 < p;\quad
h$ по пр. инд. к квадратам

$\left\{ 
 \begin{matrix}
  \alpha_1 = t_{1,1}\alpha_1 + \ldots + t_{1,p-1} \alpha_{p-1} \\
  \cdots \\
  \alpha_{p-1}' = t_{p-1,1}\alpha_1 + \ldots + t_{p-1, p-1} \alpha_{p-1} \\
  \alpha_p' = \frac{b_{1,1}}{b_{p,p}}\alpha_1 + \ldots + \frac{b_{p-1,p}}{b_{p,p}}\alpha_{p-1} + \alpha_p \\
  \alpha_{p+1}' = \alpha_{p+1} \\
  \cdots \\
  \alpha_n' = \alpha_n \\
 \end{matrix} \right.
$ \\

$T_1 = 
\begin{pmatrix}
 t_{1,1} & \cdots & t_{1,p-1} & 0 & | & \cdots & 0 \\
 \cdots & \cdots & \cdots & \cdots & | & \cdots & \cdots \\
 t_{p-1, 1} & \cdots & t_{p-1, p-1} & 0 & | & \cdots & 0 \\
 \frac{b_{1,p}}{b_{p,p}} & \cdots & \frac{b_{p-1,p}}{b_{p,p}} & 1 & | & \cdots & 0 \\
 \overline{0} & \overline{\cdots} & \overline{\cdots} & \overline{0} & 1  & \cdots & 0 \\
 \cdots & \cdots & \cdots & \cdots & \cdots &\ddots & \cdots \\
 0 & \cdots & \cdots & 0 & \cdots & \cdots & 1 
\end{pmatrix};\quad |T_1| \ne 0; T_1^{-1} $ - матр. перехода от старого к новому\\

$(\times \times) = \sum\limits_{i=1}^{p-1} \lambda_i (\alpha_i')^2 + b_{p,p} (\alpha_p')^2$

Матрица кв. формы в новом базисе: 
$\begin{pmatrix}
  \lambda_1 & \mathcal{O}\\
  \mathcal{O} & \ddots & \mathcal{O} \\
  \mathcal{O} & \cdots & \lambda_{p-1} & \mathcal{O} \\
  \mathcal{O} & \mathcal{O} & \cdots & b_{p,p} & \mathcal{O} \\
  \mathcal{O} & \mathcal{O} & \cdots & \mathcal{O} & \ddots & \mathcal{O} \\
  \mathcal{O} & \mathcal{O} & \cdots & \cdots & \mathcal{O} & \mathcal{O}
 \end{pmatrix}
$
\end{enumerate}

\end{enumerate}







