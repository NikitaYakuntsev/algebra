\subsection{Билет 23}

\textit{Обобщённая теорема Пифагора. Ортогональное дополнение. Разложение конечномерного евклидова пространства в прямую сумму подпространства и его ортогонального дополнения.}

\textbf{Теорема 1:} (Обобщенная теорема Пифагора) Пусть дана ортогональная система векторов $e_1,\ldots,e_k$ в евклидовом пространстве. $x = \sum_{i=1}^k e_i$

Тогда, $\|x\|^2 = \sum_{i=1}^k \|e_i\|^2$

\textbf{Доказательство:}

$$
  \|x\|^2 = (x,x) = (\sum_{i=1}^k e_i,\sum_{j=1}^k e_j) = \sum_{i=1}^k \sum_{j=1}^k (e_i,e_j)
$$

$$
  \forall i,j ~ i\neq j ~ (e_i,e_j) = 0
$$

$$
 \sum_{i=1}^k \sum_{j=1}^k (e_i,e_j) = \sum_{i=1}^k (e_i,e_i) = \sum_{i=1}^k\|e_i\|^2
$$ 

Пусть есть множество $M \subset E$. Ортогональным дополением к множеству $M$ называют множество $M^\bot = \{x\in E: \forall y \in M ~ (x,y) = 0\}$

\textbf{Теорема 2:} $\forall M \subset E ~ M^\bot$ - линейное подпространство.

\textbf{Доказательство:}

\begin{enumerate}
 \item $\forall x,y \in M^\bot ~ x+y \in M^\bot$
      
       $\forall z \in M ~ (x+y,z) = (x,z)+(y,z) = 0+0 = 0 \Rightarrow x+y \in M^\bot$
 \item $\forall x \in M^\bot ~ \forall \alpha \in R ~ \alpha x \in M^\bot$
       
       $\forall z \in M ~ (\alpha x,z) = \alpha(x,z) = \alpha * 0 = 0 \Rightarrow \alpha x \in M^\bot$
\end{enumerate}

\textbf{Теорема 3:} Пусть $E$ - евклидово пространство, $\dim E = n < \infty$. Пусть $L \subset E$ - линейное подпространство. Тогда, $E = L + L^\bot$

\textbf{Доказательство: } 

\begin{enumerate}
 \item $E = \{0\}$ - очевидно
 \item $\dim E = n > 0$
       \begin{enumerate}
        \item $L = 0, L^\bot = E$
        
	      $\forall x \in E ~ x = 0 + x$
	\item $L = E, L^\bot = 0$
        
	      $\forall x \in E ~ x = x + 0$
	\item $\dim L = k > 0$
	
	      Пусть $e_1,\ldots,e_k$ - базис в $L$.
	      Дополним до базиса в $E - e_1,\ldots,e_n$
	      
	      Проведем процесс ортогонализации и получим ортонормированный базис $f_1,\ldots, f_n$
	      
	      $L(f_1,\ldots,f_k) = L(e_1,\ldots,e_k) = M$
	      
	      Обозначим $L(f_{k+1},\ldots,f_n) = M$. $M = L^\bot$?
	      
	      \begin{enumerate}
	       \item $M\subset L^\bot$
	       
		     $\forall x \in M ~ x = \sum_{i=k+1}^n \alpha_i f_i$
		     
		     $\forall j = 1,\ldots,k ~ (x,f_j) = \sum_{i=k+1}^n \alpha_i (f_i,f_j) = 0$
		     
		     $x \bot (f_1,\ldots,f_k) \Rightarrow x \in L^\bot$
		     
	       \item $L^\bot \subset M$
		     
		     $\forall x \in L^\bot ~ x = \sum_{i=1}^n \alpha_i f_i$
		     
		     $\forall j = 1,\ldots,k ~ (x,f_j) = \sum_{i=1}^n \alpha_i (f_i,f_j) = \alpha_i$
		     
		     $\forall x \in L^\bot \forall j = 1,\ldots,k (x,f_j) = 0 \Rightarrow \alpha_i = 0 \Rightarrow x = \sum_{i=k+1}^n \alpha_i f_i \in M$
	      \end{enumerate}
	      $M = L^\bot \Rightarrow (f_{k+1},\ldots,f_n)$ - базис $L^\bot$.
	      
	      $\forall x \in E ~ x = \sum_{i=1}^n \alpha_i f_i = \sum_{i=1}^k \alpha_i f_i + \sum_{i=k+1}^n \alpha_i f_i = x_1 + x_2, ~ x_1 \in L, ~ x_2 \in L^\bot$
	\end{enumerate}

\end{enumerate}

