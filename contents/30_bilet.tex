\subsection{Билет 30}

\textit{Закон инерции квадратичных форм. Положительно (отрицательно) определённые квадратичные формы. Критерий Сильвестра положительной (отрицательной) определённости квадратичной формы.}\\

\textbf{Теорема 1} $E-n$-мерное вещ-ное линейное пр-во, $0 < n < \infty$

Количество 
\begin{enumerate}
 \item положительных $k$
 \item отрицательных $l$
 \item нулевых $m$
\end{enumerate} коэф-тов в каноническом виде не зависит от выбора канонического базиса

$b(x,x) = \sum\limits_{i,j=1}^n b_{i,j} \alpha_i \alpha_j;\ b_{i,j} = b(e_i, e_j),\ e=(e_1,..,e_n) - произв. базис$

При изменении базиса ранг матрицы $[b]_e$ сохраняется, \quad $[b]_{e'} = T_{e\to e'}' [b]_e T_{e \to e'} \underbrace{\Rightarrow} $\\$ |T_{e\to e'}| \ne 0$

\textbf{Лемма} При умножении матрицы на невырожд. ранг матрицы не меняется. $\mathcal{C} = \mathcal{AB};\ \mathcal{A}$ - невырожд.

\textbf{Доказательство: } $r_{AB} \le r_{B}$
\begin{enumerate}
 \item $r_{C} \le r_{B};\ A^{-1} C = A^{-1}AB = B;\ B = A^{-1}C$
 \item $r_{B} \le r_{C}$
\end{enumerate} $\Rightarrow r_B = r_C$

$\overbrace{\Rightarrow} r_{[b]_e'} = r_{[b]_e}$

\textbf{Доказательство: } 
\begin{enumerate}
 \item 
       \begin{enumerate}
        \item k
	\item l
	\item r
       \end{enumerate}
    $k+l$ равно рангу матрицы в любом базисе. 

    в базисе $f = (f_1,..,f_m)\ b(x,x) = \lambda_1 \alpha_1^2 + \ldots + \lambda_k \alpha_k^2 - \lambda_{k+1}\alpha_{k+1}^2 - \ldots - \lambda_{k+l}\alpha_{k+l}^2\quad (*)$
 \item 
      \begin{enumerate}
       \item p
       \item q
       \item r
      \end{enumerate}
    в базисе $g = (g_1,..,g_n )\  b(x,x) = \mu_1\beta_1^2 + \ldots + \mu_p\beta_p^2 - \mu_{p+1}\beta_{p+1}^2 - \ldots - \mu_{p+q} \beta_{p+q}^2 (**)$ 
\end{enumerate}

$[b]_f = 
\begin{pmatrix}
\lambda_1 \\
 & \ddots \\          
 & & \lambda_k \\
 & & & -\lambda_{k+1} \\
 & & & & \ddots \\
 & & & & & -\lambda_{k+l} \\
 & & & & & & 0 \\
 & & & & & & & \ddots \\
 & & & & & & & &0
\end{pmatrix};$

$ [b]_g = 
\begin{pmatrix}
\mu_1 \\
 & \ddots \\          
 & & \mu_p \\
 & & & -\mu_{p+1} \\
 & & & & \ddots \\
 & & & & & -\mu_{p+q} \\
 & & & & & & 0 \\
 & & & & & & & \ddots \\
 & & & & & & & &0 
\end{pmatrix}
$\\

$r_{[b]_f} = k+l;\ r_{[b]_g} = p+q \Rightarrow k+l = p+q [= m = r]$ - число ненулевых элементов
\underline{Докажем, что} $k = p, l = q$

$\begin{pmatrix} \alpha_1 \\ \cdots \\ \alpha_n \end{pmatrix} = T_{f\to e} \begin{pmatrix} \xi_1 \\ \cdots \\ \xi_n \end{pmatrix} \quad \begin{pmatrix} \beta_1 \\ \cdots \\ \beta_n \end{pmatrix} = S \begin{pmatrix} \xi_1 \\ \cdots \\ \xi_n \end{pmatrix}$

\underline{Предположим противное} $k < p:$

$\left\{ \begin{matrix} \alpha_1 = 0 \\ \cdots \\ \alpha_k = 0 \\ \beta_{p+1} = 0 \\ \cdots \\ \beta_n = 0 \end{matrix} \right.$ В системе менее $n$ уравнений, т.к. $k<p$

$\alpha_i = t_{i,1}\xi_1 + \ldots + t_{i,n}\xi_n$ - у системы $\exists$ ненул. реш. $\Rightarrow \exists x = \sum \xi_i e_i \ne 0\ b(x,x)=0$ %TODO разобраться

$(*), (**):$\quad подставим сюда нули \\
$ \lambda_1 \alpha_1^2 + \ldots + \lambda_k \alpha_k^2 - \lambda_{k+1}\alpha_{k+1}^2 - \ldots - \lambda_{k+l}\alpha_{k+l}^2 = \mu_1\beta_1^2 + \ldots + \mu_p\beta_p^2 - \mu_{p+1}\beta_{p+1}^2 - \ldots - \mu_{p+q} \beta_{p+q}^2$ 

$\Rightarrow \mu_1\beta_1^2 + \ldots + \mu_p\beta_p^2 + \lambda_{k+1}\alpha_{k+1}^2 + \ldots + \lambda_{k+l}\alpha_{k+l}^2 = 0;\ \mu_i > 0,\ \lambda_j > 0$

$\Rightarrow \beta_1 = \ldots = \beta_p = 0 = \alpha_{k+1} = \ldots = \alpha_{k+l}\ \beta_{p+1} = \ldots = \beta_n = 0$, все $\beta_i=0 \Rightarrow x=0$, но $\exists \xi_i \ne 0 \Rightarrow x \ne 0$ \underline{получили противоречие} $\Rightarrow k\ge p$, аналогично докажем что $p=k \Rightarrow q=l$

\textbf{Опр. 1} Число положительных коэф-тов в каноническом виде наз-ся положительным индексом. Число отрицательных - отрицательным индексом. Сумма кол-ва положительных и отрицательным - просто индекс (ранг).\\

\begin{center}
 \textbf{Критерий Сильвестра}
\end{center}

\textbf{Опр. }: $k(x)=b(x,x)$ кв. форма в $n$-мерном вещ. пространстве $E, k(x)$ – полож.опр. на подпр. $L\subset E$

\begin{enumerate}
 \item Кв. форма $k(x)$ - положительно (отриц.) определенная $\Leftrightarrow \forall x \in E\ k(x) \ge 0 (\le 0)$
 \item знакоопределенная $\Leftrightarrow $ положительно или отрицательно определенная
 \item знакопеременная $\Leftrightarrow \exists x', x''\ : k(x') > 0;\ k(x'') < 0$
 \item квазиположительная (отрицательная) $\Leftrightarrow \forall x\ k(x) \ge 0 (\le 0),\ \exists x \ne 0 : k(x) = 0$ 
\end{enumerate}

\textbf{Теорема} $k(x)=b(x,x)$ - полож. опр. $\Leftrightarrow \Delta_1 = b_{1,1} > 0;\ \Delta_2 = \begin{vmatrix} b_{1,1} & b_{2,2} \\ b_{2,1} & b_{2,2} \end{vmatrix} > 0 \ldots \Delta_n > 0$

\textbf{Доказательство:} %TODO

\underline{Следствие:} $k(x)=b(x,x)$ - отриц. опр. $\Leftrightarrow \left\{ \begin{matrix} \Delta_1 < 0 \\ \Delta_2 > 0 \\ \cdots \\ \mathrm{sgn}\, \Delta_n = (-1)^n \end{matrix} \right.$
                                                                                                                                                      














