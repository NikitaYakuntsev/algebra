\subsection{Билет 24}


\textit{Вычисление скалярного произведения векторов через координаты в различных базисах. Матрица Грама и её свойства. Определитель Грамма.}


\textbf{Теорема 1} \\
Пусть $E$-евклидово пространство, $0<dimE=n<\infty$ \\
$(*)(e_1..e_n)$-базис в $E, \forall x,y \in E\ , \ x=\sum \limits_{i=1}^n {\alpha_i e_i} $ \\
Тогда если:
\begin{description}
 \item [I.] (*) - ортонорм базис $y=\sum \limits_{j=1}^n {\beta_j e_j}$ \\
	то $(x,y)=\sum \limits_{i=1}^n {\alpha_i \beta_i} = (\alpha_1..\alpha_n) \begin{pmatrix} \beta_1 \\ \vdots \\ \beta_n \end{pmatrix} = [x]_e'[y]_e$
 \item [II.] (*) - произвольный базис, то $(x,y)=\sum \limits_{i=1}^n \sum \limits_{j=1^n} {\alpha_i \beta_j(e_i,e_j)} = \\ $
       $= (\alpha_1 .. \alpha_n)$ 
       $\begin{pmatrix}
         (e_1,e_1) & (e_1,e_2) & \cdots & (e_1,e_n) \\
         (e_2,e_1) & (e_2,e_2) & \cdots & (e_2,e_n) \\
         \vdots & \vdots & \ddots & \vdots \\
         (e_n,e_1) & (e_n,e_2) & \cdots & (e_n,e_n) \\
        \end{pmatrix}$
       $\begin{pmatrix}
         \beta_1 \\ \beta_2 \\ \vdots \\ \beta_n \\
        \end{pmatrix}$ = 
       $[x]_e'G(e_1..e_n)[y]_e$ 
\end{description}
\textbf{Доказательство(1)} \\
$x=\sum \limits_{i=1}^n {\alpha_i e_i}, \ y=\sum \limits_{j=1}^n {\beta_j e_j} $ \\
$(x,y)=\left( \sum \limits_{i=1}^n {\alpha_i e_i},\sum \limits_{j=1}^n {\beta_j e_j} \right) = \sum \limits_{i=1}^n \sum \limits_{j=1}^n {\alpha_i \beta_j (e_i,e_j)} $ \\
При $(e_1..e_n)$ - ортонормированный базис \\
$(e_i,e_j)=$
$\left\{ \begin{matrix}
\mbox{$0, \ i \ne j $} \\
\mbox{$1, \ i=j$}
\end{matrix}\right.$ \\
$(x,y)=\sum \limits_{i=1}^n {\alpha_i \beta_i} = (\alpha_1..\alpha_n) \begin{pmatrix} \beta_1 \\ \vdots \\ \beta_n \end{pmatrix} \Rightarrow$ (1)-доказано\\
\textbf{Доказательство(2)} \\
$((e_1,e_1),(e_1,e_2),..,(e_1,e_n)) \begin{pmatrix} \beta_1 \\ \vdots \\ \beta_n \end{pmatrix} = \sum \limits_{j=1}^n {(e_i,e_j) \beta_j}, \ i=1..n $ \\
$(\alpha_1..\alpha_n) \begin{pmatrix} \sum \limits_{j=1}^n {(e_i,e_j)\beta_1} \\ \cdots \\ \sum \limits_{j=1}^n {(e_i,e_j)\beta_j} \end{pmatrix}$ = 
$\sum \limits_{i=1}^n \sum \limits_{j=1}^n {(e_i,e_j)\alpha_i \beta_j} $ \\
\textbf{Def: } $e_1,e_2..e_n \in E\ (*)$ \\
Матрицей Грама системы векторов (*) называется матрица \\ $G(e_1..e_n)=$ 
       $\begin{pmatrix}
         (e_1,e_1) & (e_1,e_2) & \cdots & (e_1,e_n) \\
         (e_2,e_1) & (e_2,e_2) & \cdots & (e_2,e_n) \\
         \vdots & \vdots & \ddots & \vdots \\
         (e_n,e_1) & (e_n,e_2) & \cdots & (e_n,e_n) \\
        \end{pmatrix}$ \\
Определителем Грама называется $detG(e_1..e_n)$\\
\textbf{Теорема 2} \\
Система векторов $(e_1..e_n)$ 
\begin{enumerate}
 \item ортогональная $\Leftrightarrow G(e_1..e_n)$ - диагональная 
 \item ортонормированная $\Leftrightarrow G(e_1..e_n)$ - единичная 
\end{enumerate}
\textbf{Доказательство} 
\begin{enumerate}
 \item ортогональная $\Leftrightarrow (e_i,e_j)=0$ при $i \ne j \Leftrightarrow$ все внедиагональные равны 0 $\Leftrightarrow G(e_1..e_n)$ - диагональная
 \item $(e_1..e_n)$ - ортонорм $\Leftrightarrow (e_i,e_j)=$ $\left\{ \begin{matrix}
\mbox{$0, \ i \ne j $} \\
\mbox{$1, \ i=j$}
\end{matrix}\right. \Leftrightarrow G(e_1..e_n)$-единичная 
\end{enumerate}
\textbf{Теорема 3(Связь матриц Грама в разных базисах)} \\
Пусть $e=(e_1..e_n)$ - старый базис в E  $e'=(e_1'..e_n')$- новый базис в E \\
$T_{e \to e'}$ - матрица перехода \\
Пусть $G(e_1..e_n)$ и $G(e_1'..e_n')$ - матрицы Грама \\
Тогда $G(e_1'..e_n')=T_{e \to e'}' G(e_1..e_n)T_{e \to e'}$ \\
\textbf{Доказательство} \\
$\forall x,y \in E\ , \ x=\sum \limits_{i=1}^n {\alpha_i e_i} , \ y=\sum \limits_{j=1}^n {\beta_j e_j} $ \\
$(1) \ (x,y)=[x]_e'G(e_1..e_n)[y]_e$ - в старом \\
$x=\sum \limits_{i=1}^n {\alpha_i' e_i'} , \ y=\sum \limits_{j=1}^n {\beta_j' e_j'}$ \\
$(2) \ (x,y)=[x]_{e'}' G(e_1'..e_n') [y]_{e'} $ - в новом \\
$[x]_e=T_{e \to e'} [x]_{e'}$ \\
$[y]_e=T_{e \to e'} [y]_{y'}$ из (1) \\
$(x,y)=(T_{e \to e'}[x]_{e'})'G(e_1..e_n)T_{e \to e'} [y]_{e'} (*)$\\
$(AB)'=B'A'$ \\
$(*)=[x]_{e'}'(T_{e \to e'}'G(e_1..e_n)T_{e \to e'})[y]_{e'}$ \\
$G(e_1'..e_n')=T_{e \to e'}'G(e_1..e_n)T_{e \to e'}$ \\
\textbf{Теорема 4(Критерий ЛЗ и ЛНЗ)} \\
Пусть $(*)(e_1..e_n)$- произвольная система векторов в $E$- евкл. пр-ве. Тогда
\begin{enumerate}
 \item $(*)$ ЛЗ $\Leftrightarrow detG(e_1..e_n)=0$
 \item $(*)$ ЛНЗ $\Leftrightarrow detG(e_1..e_n)>0$ 
\end{enumerate}
\textbf{Доказательство} \\
\begin{enumerate}
 \item $(*)$-ЛЗ $\Rightarrow detG(e_1..e_n)=0$ ? \\
       $(*)$ ЛЗ $\Rightarrow \left[ \begin{matrix}
			      \mbox{$e_1=0$} \\
			      \mbox{$\exists j \ge 2$}
			      \end{matrix}\right.$ 
       $e_j=\sum \limits_{i=1}^{j-1} {\alpha_i e_i}$\\
       $e_1=0, \ (e_1,e_k)=0, \ k=1..n \Rightarrow$ - 1-строка $=0 \Rightarrow detG(e_1..e_n)=0$ \\
       j-столб\\
       $\begin{pmatrix}
        (e_1,e_j) \\ (e_2,e_j) \\ \cdots \\ (e_n,e_j) 
       \end{pmatrix}$ =
       $\begin{pmatrix}
         \sum \limits_{i=1}^{j-1} {(e_1,e_i)} \\
         \sum \limits_{i=1}^{j-1} {(e_2,e_i)} \\
         \cdots \\
         \sum \limits_{i=1}^{j-1} {(e_n,e_i)} \\
        \end{pmatrix}$=
        $\alpha_1 \begin{pmatrix}
                   (e_1,e_1) \\ (e_2,e_1) \\ \cdots \\ (e_n,e_1) \\
                  \end{pmatrix} +
         \alpha_2 \begin{pmatrix}
                   (e_1,e_2) \\ (e_2,e_2) \\ \cdots \\ (e_n,e_2) \\
                  \end{pmatrix} + .. +
         \alpha_n \begin{pmatrix}
                   (e_1,e_{j-1}) \\ (e_2,e_{j-1}) \\ \cdots \\ (e_n,e_{j-1})
                  \end{pmatrix}$ - линейная комбинация $\Rightarrow$ ЛЗ $\Rightarrow G(e_1..e_n)=0$  
 \item $(*)$-ЛНЗ $\Rightarrow detG(e_1..e_n)>0$ ? \\
       Существует ортонормированный базис $(e_1'..e_n')$ \\
       $\underset {=J-edin} {G(e_1'..e_n')} = T_{e \to e'}'G(e_1..e_n)T_{e \to e'}$ \\
       $\left| J \right| = \left| T_{e \to e'}'G(e_1..e_n)T_{e \to e'} \right| = \left| T_{e \to e'}' \right| \left| G(e_1..e_n) \right| \left| T_{e \to e'} \right| $ \\
       $\left| A \right| = \left| A' \right|)$ \\
       $1=\left| T_{e \to e'} \right|^2 \left| G(e_1..e_n) \right| (**)$ \\
       $T_{e \to e'}$ - невырожденная $\left| T_{e \to e'} \right| \ne 0 \Rightarrow \left| T_{e \to e'} \right|^2 > 0$\\
       $(**)\Rightarrow \left| G(e_1..e_n) \right| > 0$
 \item (В обратную сторону) $detG(e_1..e_n)=0 \Rightarrow (*)$ - ЛЗ ? \\
       \underline{Предположим противное} $(*)$ -ЛНЗ $\stackrel{2)}{\Rightarrow} detG(e_1..e_n)>0$
 \item $detG(e_1..e_n)>0 \Rightarrow (*)$ -ЛНЗ ? \\
       \underline{Предположим противное} $(*)$ -ЛЗ $\stackrel{1)}{\Rightarrow} detG(e_1..e_n)=0$
\end{enumerate}





        

