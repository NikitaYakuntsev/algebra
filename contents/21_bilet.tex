\subsection{Билет 21}

\textit{Норма. Линейное нормированное пространство. Метрика. Метрическое пространство.}

Пусть $E$ - линейное пространство над полем $R$ (или $C$). 

Функция $f:E\to R: x \in E \to \| x \| \in R$ называется нормой, если:
\begin{enumerate}
 \item \begin{enumerate}
        \item $\|x\| \ge 0$
        \item $\|x\| = 0 \Leftrightarrow x = 0$
       \end{enumerate}
 \item $\forall x \in E ~ \forall \alpha \in R ~ \|\alpha x \| = \left| \alpha \right| \|x\|$
 \item Неравенство треугольника: $\|x+y\| \le \|x\| + \|y\|$
\end{enumerate}

При этом, $E$ - линейное нормированное пространство.

Примеры:

\begin{enumerate}
 \item $E = R: \forall x \in R \|x\| = \left| x \right|$
 \item $E = V^3: \|x\| = \sqrt{(x,x)}$
 \item $E = R^n: x = \left(\begin{matrix}
                            \mbox{$\xi_1$}\\
                            \mbox{...}\\
                            \mbox{$\xi_n$}
                           \end{matrix}\right) \|x\| = \max \xi_i$ hay

\end{enumerate}

